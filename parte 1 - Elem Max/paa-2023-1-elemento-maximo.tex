\documentclass[12pt]{article}
\usepackage[utf8]{inputenc}
%\usepackage[portuguese]{babel}
\usepackage{amsmath,amsfonts}
\usepackage{coqdoc}
\usepackage{mathpartir,proof}
\usepackage{minibox}
\usepackage{tabto,calc}
\usepackage{graphicx}

% \Def\figurename{Figura}
\def\refname{Refer\^{e}ncias}
\def\bibname{Refer\^{e}ncias}

\newtheorem{teorema}{Teorema}
\newtheorem{definicao}[teorema]{Definição}
\newtheorem{exercicio}[teorema]{Exercício}
\newtheorem{exemplo}[teorema]{Exemplo}
\newtheorem{lema}[teorema]{Lema}
\newtheorem{corolario}[teorema]{Corolário}

\newcommand{\nat}{\mathbb{N}}
\newcommand{\tto}{\Longrightarrow}
\newcommand{\toot}{\leftrightarrow}
\newcommand{\rweak}{(R_w)}
\newcommand{\ax}{\rm{(Ax)}}
\newcommand{\refl}{\rm{(Refl)}}
\newcommand{\lem}{\rm{(LEM)}}
\newcommand{\mt}{\rm{(MT)}}
\newcommand{\cp}{\rm{(CP)}}
\newcommand{\lp}{\rm{(LP)}}
\newcommand{\ror}{(R_\lor)}
\newcommand{\rto}{(R_\to)}
\newcommand{\lto}{(L_\to)}
\newcommand{\rcont}{(R_c)}
\newcommand{\toe}{(\to_e)}
\newcommand{\toi}[1]{(\to_i)\;{#1}}
\newcommand{\toisa}{\rm{(\to_i)}}
\newcommand{\landi}{(\land_i)}
\newcommand{\lande}{(\land_e)}
\newcommand{\landel}{(\land_{e_1})}
\newcommand{\lander}{(\land_{e_2})}
\newcommand{\lori}{(\lor_i)}
\newcommand{\lorl}{(\lor_{i_1})}
\newcommand{\lorr}{(\lor_{i_2})}
\newcommand{\lore}[2]{(\lor_e)\;{#1,#2}}
\newcommand{\loresa}{\rm{(\lor_e)}}
\newcommand{\nege}{(\neg_e)}
\newcommand{\nne}{(\neg\neg_e)}
\newcommand{\nni}{(\neg\neg_i)}
\newcommand{\negi}[1]{(\neg_i)\;{#1}}
\newcommand{\negisa}{\rm{(\neg_i)}}
\newcommand{\pbc}[1]{(PBC)\;{#1}}
\newcommand{\pbcsa}{\rm{(PBC)}}
\newcommand{\bote}{(\bot_e)}
\newcommand{\alle}{(\forall_e)}
\newcommand{\alli}{(\forall_i)}
\newcommand{\exie}[1]{(\exists_e)\;{#1}}
\newcommand{\exii}{(\exists_i)}

\begin{document}
\bibliographystyle{alpha}
\begin{center}
  \Large Título do Relatório \\ 
  \normalsize \today \\ 
  Coloque aqui o seu nome
\end{center}

\begin{coqdoccode}
\end{coqdoccode}
\section{Algoritmo que retorna o maior elemento de uma lista de naturais.}

\begin{coqdoccode}
\coqdocemptyline
\coqdocnoindent
\coqdockw{Require} \coqdockw{Import} \coqdocvar{Arith} \coqdocvar{List} \coqdocvar{Lia}.\coqdoceol
\coqdocemptyline
\coqdocemptyline
\coqdocnoindent
\coqdockw{Fixpoint} \coqdocvar{elt\_max\_aux} \coqdocvar{l} \coqdocvar{max} :=\coqdoceol
\coqdocindent{1.00em}
\coqdockw{match} \coqdocvar{l} \coqdockw{with}\coqdoceol
\coqdocindent{1.00em}
\ensuremath{|} \coqdocvar{nil} \ensuremath{\Rightarrow} \coqdocvar{max}\coqdoceol
\coqdocindent{1.00em}
\ensuremath{|} \coqdocvar{h}::\coqdocvar{tl} \ensuremath{\Rightarrow} \coqdockw{if} \coqdocvar{max} \ensuremath{<?} \coqdocvar{h} \coqdockw{then} \coqdocvar{elt\_max\_aux} \coqdocvar{tl} \coqdocvar{h} \coqdockw{else} \coqdocvar{elt\_max\_aux} \coqdocvar{tl} \coqdocvar{max}\coqdoceol
\coqdocindent{1.00em}
\coqdockw{end}.\coqdoceol
\coqdocemptyline
\coqdocnoindent
\coqdockw{Eval} \coqdoctac{compute} \coqdoctac{in} (\coqdocvar{elt\_max\_aux} (1::2::3::\coqdocvar{nil}) 0).\coqdoceol
\coqdocnoindent
\coqdockw{Eval} \coqdoctac{compute} \coqdoctac{in} (\coqdocvar{elt\_max\_aux} (1::2::3::\coqdocvar{nil}) 7).\coqdoceol
\coqdocemptyline
\coqdocemptyline
\coqdocnoindent
\coqdockw{Definition} \coqdocvar{ge\_all} \coqdocvar{x} \coqdocvar{l} := \coqdockw{\ensuremath{\forall}} \coqdocvar{y}, \coqdocvar{In} \coqdocvar{y} \coqdocvar{l} \ensuremath{\rightarrow} \coqdocvar{y} \ensuremath{\le} \coqdocvar{x}.\coqdoceol
\coqdocnoindent
\coqdockw{Infix} ">=*" := \coqdocvar{ge\_all} (\coqdoctac{at} \coqdockw{level} 70, \coqdockw{no} \coqdockw{associativity}).\coqdoceol
\coqdocemptyline
\coqdocemptyline
\coqdocnoindent
\coqdockw{Definition} \coqdocvar{le\_all} \coqdocvar{x} \coqdocvar{l} := \coqdockw{\ensuremath{\forall}} \coqdocvar{y}, \coqdocvar{In} \coqdocvar{y} \coqdocvar{l} \ensuremath{\rightarrow} \coqdocvar{x} \ensuremath{\le} \coqdocvar{y}.\coqdoceol
\coqdocnoindent
\coqdockw{Infix} "<=*" := \coqdocvar{le\_all} (\coqdoctac{at} \coqdockw{level} 70, \coqdockw{no} \coqdockw{associativity}).\coqdoceol
\coqdocemptyline
\coqdocnoindent
\coqdockw{Lemma} \coqdocvar{elt\_max\_aux\_large}: \coqdockw{\ensuremath{\forall}} \coqdocvar{l} \coqdocvar{a}, \coqdocvar{a} \ensuremath{\le} \coqdocvar{elt\_max\_aux} \coqdocvar{l} \coqdocvar{a}.\coqdoceol
\coqdocnoindent
\coqdockw{Proof}.\coqdoceol
\coqdocindent{1.00em}
\coqdoctac{induction} \coqdocvar{l}.\coqdoceol
\coqdocindent{1.00em}
- \coqdoctac{intro} \coqdocvar{a}. \coqdoctac{simpl}. \coqdocvar{lia}.\coqdoceol
\coqdocindent{1.00em}
- \coqdoctac{intro} \coqdocvar{a'}. \coqdoctac{simpl}. \coqdoctac{destruct} (\coqdocvar{a'} \ensuremath{<?} \coqdocvar{a}) \coqdocvar{eqn}:\coqdocvar{H}.\coqdoceol
\coqdocindent{2.00em}
+ \coqdoctac{apply} \coqdocvar{Nat.le\_trans} \coqdockw{with} \coqdocvar{a}.\coqdoceol
\coqdocindent{3.00em}
\ensuremath{\times} \coqdoctac{apply} \coqdocvar{Nat.ltb\_lt} \coqdoctac{in} \coqdocvar{H}. \coqdoctac{apply} \coqdocvar{Nat.lt\_le\_incl}; \coqdoctac{assumption}.\coqdoceol
\coqdocindent{3.00em}
\ensuremath{\times} \coqdoctac{apply} \coqdocvar{IHl}.\coqdoceol
\coqdocindent{2.00em}
+ \coqdoctac{apply} \coqdocvar{IHl}.\coqdoceol
\coqdocnoindent
\coqdockw{Qed}.\coqdoceol
\coqdocemptyline
\coqdocnoindent
\coqdockw{Lemma} \coqdocvar{elt\_max\_aux\_le}: \coqdockw{\ensuremath{\forall}} \coqdocvar{l} \coqdocvar{a} \coqdocvar{a'}, \coqdocvar{a} \ensuremath{\le} \coqdocvar{a'} \ensuremath{\rightarrow}  \coqdocvar{elt\_max\_aux} \coqdocvar{l} \coqdocvar{a} \ensuremath{\le} \coqdocvar{elt\_max\_aux} \coqdocvar{l} \coqdocvar{a'}.\coqdoceol
\coqdocnoindent
\coqdockw{Proof}.\coqdoceol
\coqdocindent{1.00em}
\coqdoctac{induction} \coqdocvar{l}.\coqdoceol
\coqdocindent{1.00em}
- \coqdoctac{intros} \coqdocvar{a} \coqdocvar{a'} \coqdocvar{H}. \coqdoctac{simpl}. \coqdoctac{assumption}.\coqdoceol
\coqdocindent{1.00em}
- \coqdoctac{intros} \coqdocvar{a'} \coqdocvar{a'{}'} \coqdocvar{H}. \coqdoctac{simpl}. \coqdoctac{destruct} (\coqdocvar{a'} \ensuremath{<?} \coqdocvar{a}) \coqdocvar{eqn}:\coqdocvar{H1}.\coqdoceol
\coqdocindent{2.00em}
+ \coqdoctac{destruct} (\coqdocvar{a'{}'} \ensuremath{<?} \coqdocvar{a}) \coqdocvar{eqn}:\coqdocvar{H2}.\coqdoceol
\coqdocindent{3.00em}
\ensuremath{\times} \coqdocvar{lia}.\coqdoceol
\coqdocindent{3.00em}
\ensuremath{\times} \coqdoctac{apply} \coqdocvar{IHl}. \coqdoctac{apply} \coqdocvar{Nat.ltb\_ge}. \coqdoctac{assumption}.\coqdoceol
\coqdocindent{2.00em}
+ \coqdoctac{destruct} (\coqdocvar{a'{}'} \ensuremath{<?} \coqdocvar{a}) \coqdocvar{eqn}:\coqdocvar{H2}.\coqdoceol
\coqdocindent{3.00em}
\ensuremath{\times} \coqdoctac{apply} \coqdocvar{Nat.ltb\_lt} \coqdoctac{in} \coqdocvar{H2}. \coqdoctac{apply} \coqdocvar{Nat.ltb\_ge} \coqdoctac{in} \coqdocvar{H1}. \coqdocvar{lia}.\coqdoceol
\coqdocindent{3.00em}
\ensuremath{\times} \coqdoctac{apply} \coqdocvar{IHl}. \coqdoctac{assumption}.\coqdoceol
\coqdocnoindent
\coqdockw{Qed}.\coqdoceol
\coqdocemptyline
\coqdocnoindent
\coqdockw{Lemma} \coqdocvar{elt\_max\_aux\_correct\_1}: \coqdockw{\ensuremath{\forall}} \coqdocvar{l} \coqdocvar{a}, \coqdocvar{elt\_max\_aux} \coqdocvar{l} \coqdocvar{a} \ensuremath{>=*} \coqdocvar{a}::\coqdocvar{l}.\coqdoceol
\coqdocnoindent
\coqdockw{Proof}.\coqdoceol
\coqdocindent{1.00em}
\coqdoctac{induction} \coqdocvar{l}.\coqdoceol
\coqdocindent{1.00em}
- \coqdoctac{intro} \coqdocvar{a}. \coqdoctac{simpl}. \coqdoctac{unfold} \coqdocvar{ge\_all}. \coqdoctac{intros} \coqdocvar{y} \coqdocvar{H}. \coqdoctac{inversion} \coqdocvar{H}.\coqdoceol
\coqdocindent{2.00em}
+ \coqdoctac{subst}. \coqdocvar{lia}.\coqdoceol
\coqdocindent{2.00em}
+ \coqdoctac{inversion} \coqdocvar{H0}.\coqdoceol
\coqdocindent{1.00em}
- \coqdoctac{intros} \coqdocvar{a'}. \coqdoctac{simpl}. \coqdoctac{destruct} (\coqdocvar{a'} \ensuremath{<?} \coqdocvar{a}) \coqdocvar{eqn}:\coqdocvar{H}.\coqdoceol
\coqdocindent{2.00em}
+ \coqdoctac{unfold} \coqdocvar{ge\_all} \coqdoctac{in} *. \coqdoctac{intros} \coqdocvar{y} \coqdocvar{H'}. \coqdoctac{inversion} \coqdocvar{H'}; \coqdoctac{subst}.\coqdoceol
\coqdocindent{3.00em}
\ensuremath{\times} \coqdoctac{apply} \coqdocvar{Nat.le\_trans} \coqdockw{with} \coqdocvar{a}.\coqdoceol
\coqdocindent{4.00em}
** \coqdoctac{apply} \coqdocvar{Nat.ltb\_lt} \coqdoctac{in} \coqdocvar{H}. \coqdoctac{apply} \coqdocvar{Nat.lt\_le\_incl}; \coqdoctac{assumption}.\coqdoceol
\coqdocindent{4.00em}
** \coqdoctac{apply} \coqdocvar{elt\_max\_aux\_large}.\coqdoceol
\coqdocindent{3.00em}
\ensuremath{\times} \coqdoctac{apply} \coqdocvar{IHl}; \coqdoctac{assumption}.\coqdoceol
\coqdocindent{2.00em}
+ \coqdoctac{unfold} \coqdocvar{ge\_all} \coqdoctac{in} *. \coqdoctac{intros} \coqdocvar{y} \coqdocvar{H'}. \coqdoctac{inversion} \coqdocvar{H'}; \coqdoctac{subst}.\coqdoceol
\coqdocindent{3.00em}
\ensuremath{\times} \coqdoctac{apply} \coqdocvar{elt\_max\_aux\_large}.\coqdoceol
\coqdocindent{3.00em}
\ensuremath{\times} \coqdoctac{apply} \coqdocvar{Nat.le\_trans} \coqdockw{with} (\coqdocvar{elt\_max\_aux} \coqdocvar{l} \coqdocvar{a}).\coqdoceol
\coqdocindent{4.00em}
** \coqdoctac{apply} \coqdocvar{IHl}; \coqdoctac{assumption}.\coqdoceol
\coqdocindent{4.00em}
** \coqdoctac{apply} \coqdocvar{elt\_max\_aux\_le}. \coqdoctac{apply} \coqdocvar{Nat.ltb\_ge}. \coqdoctac{assumption}.\coqdoceol
\coqdocnoindent
\coqdockw{Qed}.\coqdoceol
\coqdocemptyline
\coqdocnoindent
\coqdockw{Lemma} \coqdocvar{elt\_max\_aux\_head}: \coqdockw{\ensuremath{\forall}} \coqdocvar{l} \coqdocvar{d}, \coqdocvar{In} (\coqdocvar{elt\_max\_aux} (\coqdocvar{d}::\coqdocvar{l}) \coqdocvar{d}) (\coqdocvar{d}::\coqdocvar{l}).\coqdoceol
\coqdocnoindent
\coqdockw{Proof}.\coqdoceol
\coqdocindent{1.00em}
\coqdoctac{induction} \coqdocvar{l}.\coqdoceol
\coqdocindent{1.00em}
- \coqdoctac{intro} \coqdocvar{d}. \coqdoctac{simpl}. \coqdoctac{destruct} (\coqdocvar{d} \ensuremath{<?} \coqdocvar{d}).\coqdoceol
\coqdocindent{2.00em}
+ \coqdoctac{left}. \coqdoctac{reflexivity}.\coqdoceol
\coqdocindent{2.00em}
+ \coqdoctac{left}. \coqdoctac{reflexivity}.\coqdoceol
\coqdocindent{1.00em}
- \coqdoctac{intro} \coqdocvar{d}. \coqdoctac{simpl} \coqdoctac{in} *. \coqdoctac{assert} (\coqdocvar{H} := \coqdocvar{IHl}). \coqdoctac{specialize} (\coqdocvar{H} \coqdocvar{d}). \coqdoctac{destruct} (\coqdocvar{d} \ensuremath{<?} \coqdocvar{d}) \coqdocvar{eqn}:\coqdocvar{Hd}.\coqdoceol
\coqdocindent{2.00em}
+ \coqdoctac{rewrite} \coqdocvar{Nat.ltb\_irrefl} \coqdoctac{in} \coqdocvar{Hd}. \coqdoctac{inversion} \coqdocvar{Hd}.\coqdoceol
\coqdocindent{2.00em}
+ \coqdoctac{destruct} \coqdocvar{H}.\coqdoceol
\coqdocindent{3.00em}
\ensuremath{\times} \coqdoctac{destruct} (\coqdocvar{d} \ensuremath{<?} \coqdocvar{a}) \coqdocvar{eqn}:\coqdocvar{Ha}.\coqdoceol
\coqdocindent{4.00em}
** \coqdoctac{specialize} (\coqdocvar{IHl} \coqdocvar{a}). \coqdoctac{rewrite} \coqdocvar{Nat.ltb\_irrefl} \coqdoctac{in} \coqdocvar{IHl}. \coqdoctac{destruct} \coqdocvar{IHl}.\coqdoceol
\coqdocindent{5.50em}
*** \coqdoctac{right}. \coqdoctac{left}. \coqdoctac{assumption}.\coqdoceol
\coqdocindent{5.50em}
*** \coqdoctac{right}. \coqdoctac{right}. \coqdoctac{assumption}.\coqdoceol
\coqdocindent{4.00em}
** \coqdoctac{left}. \coqdoctac{assumption}.\coqdoceol
\coqdocindent{3.00em}
\ensuremath{\times} \coqdoctac{destruct} (\coqdocvar{d} \ensuremath{<?} \coqdocvar{a}) \coqdocvar{eqn}:\coqdocvar{Ha}.\coqdoceol
\coqdocindent{4.00em}
** \coqdoctac{specialize} (\coqdocvar{IHl} \coqdocvar{a}). \coqdoctac{rewrite} \coqdocvar{Nat.ltb\_irrefl} \coqdoctac{in} \coqdocvar{IHl}. \coqdoctac{destruct} \coqdocvar{IHl}.\coqdoceol
\coqdocindent{5.50em}
*** \coqdoctac{right}. \coqdoctac{left}. \coqdoctac{assumption}.\coqdoceol
\coqdocindent{5.50em}
*** \coqdoctac{right}. \coqdoctac{right}. \coqdoctac{assumption}.\coqdoceol
\coqdocindent{4.00em}
** \coqdoctac{right}. \coqdoctac{right}. \coqdoctac{assumption}.\coqdoceol
\coqdocnoindent
\coqdockw{Qed}.\coqdoceol
\coqdocemptyline
\coqdocnoindent
\coqdockw{Lemma} \coqdocvar{elt\_max\_aux\_correct\_2}: \coqdockw{\ensuremath{\forall}} \coqdocvar{l} \coqdocvar{d}, \coqdocvar{elt\_max\_aux} (\coqdocvar{d}::\coqdocvar{l}) \coqdocvar{d} \ensuremath{>=*} \coqdocvar{d}::\coqdocvar{l}.\coqdoceol
\coqdocnoindent
\coqdockw{Proof}.\coqdoceol
\coqdocindent{1.00em}
\coqdoctac{intros} \coqdocvar{l} \coqdocvar{d}. \coqdoctac{simpl}. \coqdoctac{rewrite} \coqdocvar{Nat.ltb\_irrefl}. \coqdoctac{apply} \coqdocvar{elt\_max\_aux\_correct\_1}.\coqdoceol
\coqdocnoindent
\coqdockw{Qed}.\coqdoceol
\coqdocemptyline
\coqdocnoindent
\coqdockw{Lemma} \coqdocvar{in\_swap}: \coqdockw{\ensuremath{\forall}} (\coqdocvar{l}: \coqdocvar{list} \coqdocvar{nat}) \coqdocvar{x} \coqdocvar{y} \coqdocvar{z}, \coqdocvar{In} \coqdocvar{z} (\coqdocvar{x}::\coqdocvar{y}::\coqdocvar{l}) \ensuremath{\rightarrow} \coqdocvar{In} \coqdocvar{z} (\coqdocvar{y}::\coqdocvar{x}::\coqdocvar{l}).\coqdoceol
\coqdocnoindent
\coqdockw{Proof}.\coqdoceol
\coqdocindent{1.00em}
\coqdoctac{intros} \coqdocvar{l} \coqdocvar{x} \coqdocvar{y} \coqdocvar{z} \coqdocvar{H}. \coqdoctac{simpl} \coqdoctac{in} *. \coqdoctac{rewrite} \ensuremath{\leftarrow} \coqdocvar{or\_assoc} \coqdoctac{in} *. \coqdoctac{destruct} \coqdocvar{H}.\coqdoceol
\coqdocindent{1.00em}
- \coqdoctac{left}. \coqdoctac{apply} \coqdocvar{or\_comm}. \coqdoctac{assumption}.\coqdoceol
\coqdocindent{1.00em}
- \coqdoctac{right}. \coqdoctac{assumption}.\coqdoceol
\coqdocnoindent
\coqdockw{Qed}.\coqdoceol
\coqdocemptyline
\coqdocnoindent
\coqdockw{Lemma} \coqdocvar{elt\_max\_aux\_in}: \coqdockw{\ensuremath{\forall}} \coqdocvar{l} \coqdocvar{x}, \coqdocvar{In} (\coqdocvar{elt\_max\_aux} \coqdocvar{l} \coqdocvar{x}) (\coqdocvar{x}::\coqdocvar{l}).\coqdoceol
\coqdocnoindent
\coqdockw{Proof}.\coqdoceol
\coqdocindent{1.00em}
\coqdoctac{induction} \coqdocvar{l}.\coqdoceol
\coqdocindent{1.00em}
- \coqdoctac{intro} \coqdocvar{x}. \coqdoctac{simpl}. \coqdoctac{left}; \coqdoctac{reflexivity}.\coqdoceol
\coqdocindent{1.00em}
- \coqdoctac{intro} \coqdocvar{x}. \coqdoctac{simpl} \coqdocvar{elt\_max\_aux}. \coqdoctac{destruct} (\coqdocvar{x} \ensuremath{<?} \coqdocvar{a}) \coqdocvar{eqn}: \coqdocvar{Hlt}.\coqdoceol
\coqdocindent{3.00em}
\ensuremath{\times} \coqdoctac{specialize} (\coqdocvar{IHl} \coqdocvar{a}). \coqdoctac{apply} \coqdocvar{in\_cons}; \coqdoctac{assumption}.\coqdoceol
\coqdocindent{3.00em}
\ensuremath{\times} \coqdoctac{specialize} (\coqdocvar{IHl} \coqdocvar{x}). \coqdoctac{apply} \coqdocvar{in\_swap}. \coqdoctac{apply} \coqdocvar{in\_cons}. \coqdoctac{assumption}.\coqdoceol
\coqdocnoindent
\coqdockw{Qed}.\coqdoceol
\coqdocemptyline
\coqdocnoindent
\coqdockw{Lemma} \coqdocvar{ge\_ge\_all}: \coqdockw{\ensuremath{\forall}} \coqdocvar{l} \coqdocvar{x} \coqdocvar{y}, \coqdocvar{x} \ensuremath{\ge} \coqdocvar{y} \ensuremath{\rightarrow} \coqdocvar{y} \ensuremath{>=*} \coqdocvar{l} \ensuremath{\rightarrow} \coqdocvar{x} \ensuremath{>=*} \coqdocvar{l}.\coqdoceol
\coqdocnoindent
\coqdockw{Proof}.\coqdoceol
\coqdocindent{1.00em}
\coqdoctac{induction} \coqdocvar{l}.\coqdoceol
\coqdocindent{1.00em}
- \coqdoctac{intros} \coqdocvar{x} \coqdocvar{y} \coqdocvar{H1} \coqdocvar{H2}. \coqdoctac{unfold} \coqdocvar{ge\_all} \coqdoctac{in} *. \coqdoctac{intros} \coqdocvar{y'} \coqdocvar{H'}. \coqdoctac{inversion} \coqdocvar{H'}.\coqdoceol
\coqdocindent{1.00em}
- \coqdoctac{intros} \coqdocvar{x} \coqdocvar{y} \coqdocvar{H1} \coqdocvar{H2}. \coqdoctac{unfold} \coqdocvar{ge\_all} \coqdoctac{in} *. \coqdoctac{intros} \coqdocvar{y'} \coqdocvar{H'}. \coqdoctac{apply} \coqdocvar{H2} \coqdoctac{in} \coqdocvar{H'}. \coqdoctac{unfold} \coqdocvar{ge} \coqdoctac{in} \coqdocvar{H1}. \coqdoctac{apply} \coqdocvar{Nat.le\_trans} \coqdockw{with} \coqdocvar{y}; \coqdoctac{assumption}.\coqdoceol
\coqdocnoindent
\coqdockw{Qed}.\coqdoceol
\coqdocemptyline
\coqdocnoindent
\coqdockw{Lemma} \coqdocvar{elt\_max\_aux\_correto}: \coqdockw{\ensuremath{\forall}} \coqdocvar{l} \coqdocvar{x}, \coqdocvar{In} \coqdocvar{x} \coqdocvar{l} \ensuremath{\rightarrow}  \coqdocvar{elt\_max\_aux} \coqdocvar{l} \coqdocvar{x} \ensuremath{>=*} \coqdocvar{l} \ensuremath{\land} \coqdocvar{In} (\coqdocvar{elt\_max\_aux} \coqdocvar{l} \coqdocvar{x}) \coqdocvar{l}.\coqdoceol
\coqdocnoindent
\coqdockw{Proof}.\coqdoceol
\coqdocindent{1.00em}
\coqdoctac{induction} \coqdocvar{l}.\coqdoceol
\coqdocindent{1.00em}
- \coqdoctac{intros} \coqdocvar{x} \coqdocvar{H}. \coqdoctac{inversion} \coqdocvar{H}.\coqdoceol
\coqdocindent{1.00em}
- \coqdoctac{intros} \coqdocvar{x} \coqdocvar{H}. \coqdoctac{inversion} \coqdocvar{H}.\coqdoceol
\coqdocindent{2.00em}
+ \coqdoctac{subst}. \coqdoctac{split}.\coqdoceol
\coqdocindent{3.00em}
\ensuremath{\times} \coqdoctac{apply} \coqdocvar{elt\_max\_aux\_correct\_2}.\coqdoceol
\coqdocindent{3.00em}
\ensuremath{\times} \coqdoctac{apply} \coqdocvar{elt\_max\_aux\_head}.\coqdoceol
\coqdocindent{2.00em}
+ \coqdoctac{clear} \coqdocvar{H}. \coqdoctac{split}.\coqdoceol
\coqdocindent{3.00em}
\ensuremath{\times} \coqdoctac{simpl}. \coqdoctac{destruct} (\coqdocvar{x} \ensuremath{<?} \coqdocvar{a}) \coqdocvar{eqn}:\coqdocvar{Hlt}.\coqdoceol
\coqdocindent{4.00em}
** \coqdoctac{apply} \coqdocvar{elt\_max\_aux\_correct\_1}.\coqdoceol
\coqdocindent{4.00em}
** \coqdoctac{replace} (\coqdocvar{elt\_max\_aux} \coqdocvar{l} \coqdocvar{x}) \coqdockw{with} (\coqdocvar{elt\_max\_aux} (\coqdocvar{a}::\coqdocvar{l}) \coqdocvar{x}).\coqdoceol
\coqdocindent{5.50em}
*** \coqdoctac{apply} \coqdocvar{ge\_ge\_all} \coqdockw{with} (\coqdocvar{elt\_max\_aux} (\coqdocvar{a}::\coqdocvar{l}) \coqdocvar{a}).\coqdoceol
\coqdocindent{7.50em}
**** \coqdoctac{apply} \coqdocvar{Nat.ltb\_ge} \coqdoctac{in} \coqdocvar{Hlt}. \coqdoctac{unfold} \coqdocvar{ge} \coqdoctac{in} *. \coqdoctac{apply} \coqdocvar{elt\_max\_aux\_le}. \coqdoctac{assumption}.\coqdoceol
\coqdocindent{7.50em}
**** \coqdoctac{apply} \coqdocvar{elt\_max\_aux\_correct\_2}.\coqdoceol
\coqdocindent{5.50em}
*** \coqdoctac{simpl}. \coqdoctac{rewrite} \coqdocvar{Hlt}. \coqdoctac{reflexivity}.\coqdoceol
\coqdocindent{3.00em}
\ensuremath{\times} \coqdoctac{simpl} \coqdocvar{elt\_max\_aux}. \coqdoctac{destruct} (\coqdocvar{x} \ensuremath{<?} \coqdocvar{a}) \coqdocvar{eqn}:\coqdocvar{Hlt}.\coqdoceol
\coqdocindent{4.00em}
** \coqdoctac{apply} \coqdocvar{elt\_max\_aux\_in}.\coqdoceol
\coqdocindent{4.00em}
** \coqdoctac{apply} \coqdocvar{in\_cons}. \coqdoctac{apply} \coqdocvar{IHl}. \coqdoctac{assumption}.\coqdoceol
\coqdocnoindent
\coqdockw{Qed}.\coqdoceol
\coqdocemptyline
\end{coqdoccode}


Agora podemos definir a função principal:


 \begin{coqdoccode}
\coqdocemptyline
\coqdocnoindent
\coqdockw{Definition} \coqdocvar{elt\_max} (\coqdocvar{l}: \coqdocvar{list} \coqdocvar{nat}) :=\coqdoceol
\coqdocindent{1.00em}
\coqdockw{match} \coqdocvar{l} \coqdockw{with}\coqdoceol
\coqdocindent{1.00em}
\ensuremath{|} \coqdocvar{nil} \ensuremath{\Rightarrow} \coqdocvar{None}\coqdoceol
\coqdocindent{1.00em}
\ensuremath{|} \coqdocvar{h}::\coqdocvar{tl} \ensuremath{\Rightarrow} \coqdocvar{Some} (\coqdocvar{elt\_max\_aux} \coqdocvar{tl} \coqdocvar{h})\coqdoceol
\coqdocindent{1.00em}
\coqdockw{end}.\coqdoceol
\coqdocemptyline
\coqdocemptyline
\coqdocnoindent
\coqdockw{Theorem} \coqdocvar{elt\_max\_correto}: \coqdockw{\ensuremath{\forall}} \coqdocvar{l} \coqdocvar{k}, \coqdocvar{elt\_max} \coqdocvar{l} = \coqdocvar{Some} \coqdocvar{k} \ensuremath{\rightarrow} \coqdocvar{k} \ensuremath{>=*} \coqdocvar{l} \ensuremath{\land} \coqdocvar{In} \coqdocvar{k} \coqdocvar{l}.\coqdoceol
\coqdocnoindent
\coqdockw{Proof}.\coqdoceol
\coqdocindent{1.00em}
\coqdoctac{induction} \coqdocvar{l}.\coqdoceol
\coqdocindent{1.00em}
- \coqdoctac{intros} \coqdocvar{k} \coqdocvar{H}. \coqdocindent{2.00em}
+ \coqdoctac{subst}. \coqdoctac{split}.\coqdoceol
\coqdocindent{3.00em}
\ensuremath{\times} \coqdoctac{unfold} \coqdocvar{ge\_all} \coqdoctac{in} *. \coqdoctac{intros} \coqdocvar{y'} \coqdocvar{H'}. \coqdoctac{inversion} \coqdocvar{H'}.\coqdoceol
\coqdocindent{3.00em}
\ensuremath{\times} \coqdoctac{inversion} \coqdocvar{H}.\coqdoceol
\coqdocindent{1.00em}
- \coqdoctac{intros} \coqdocvar{k} \coqdocvar{H}. \coqdoctac{inversion} \coqdocvar{H}.\coqdoceol
\coqdocindent{2.00em}
+ \coqdoctac{subst}. \coqdoctac{split}.\coqdoceol
\coqdocindent{3.00em}
\ensuremath{\times} \coqdoctac{apply} \coqdocvar{elt\_max\_aux\_correct\_1}.\coqdoceol
\coqdocindent{3.00em}
\ensuremath{\times} \coqdoctac{apply} \coqdocvar{elt\_max\_aux\_in}.\coqdoceol
\coqdocnoindent
\coqdockw{Qed}.\coqdoceol
\end{coqdoccode}


% Para gerar o relatório: 
% 1.Em um terminal digite (no diretório que contém o arquivo Coq): coqdoc --latex --body-only -s -g paa_2023_1_elemento_maximo.v 
%  2. Em um terminal digite: pdflatex relatorio.tex 
%  3. Abra o arquivo relatorio.pdf com o aplicativo de sua preferência.

\begin{thebibliography}{CLRS01}  
\bibitem[ARdM17]{ARdM2017}
M.~Ayala-Rinc\'on and F.L.C. de Moura.
\newblock{\em {Applied Logic for Computer Scientists
- computational deduction and formal proofs}}.
\newblock UTiCS, Springer, 2017.

\bibitem[CLRS09]{CoLeRiSt2009}
T.~H. Cormen, C.~E. Leiserson, R.~L. Rivest, and C.~Stein.
\newblock {\em {Introduction to Algorithms}}.
\newblock MIT Electrical Engineering and Computer Science Series. MIT press,
  third edition, 2009.
\end{thebibliography}

\end{document}

%%% Local Variables:
%%% mode: latex
%%% TeX-master: t
%%% End:
